\section{Gauss-Newton Optimization}
\label{sec:optimization}
$n$ variables\\
$k$ residuals\\
$\mathbf{x}$ unknown\\
Traditional Gauss-Newton definition
$$E (\mathbf{x}) = || \mathbf{F}(\mathbf{x}) ||_2^2,\qquad \mathbf{F}(\mathbf{x})  =[f_1(\mathbf{x}), \hdots, f_k(\mathbf{x})]^T$$



The optimal parameters $\mathbf{x}^{*}$ are obtained by solving the minimization problem
$$ \mathbf{x}^{*} = \displaystyle \argmin_\mathbf{x} \Vert \mathbf{F} (\mathbf{x}) \Vert_2^2.$$
%
We linearize the vector field $\mathbf{F}(\mathbf{x})$ using the first-order Taylor expansion,
and obtain a linearized solution $\mathbf{x}_{k+1}$ for a given $\mathbf{x}_k$:
%and obtain an iterative update $\mathbf{x}_k \rightarrow \mathbf{x}_{k+1}$:
$$ \mathbf{F}(\mathbf{x}_{k+1}) \approx \mathbf{F}(\mathbf{x}_{k})+\mathbf{J}(\mathbf{x}_{k})\delta_k,\qquad \delta_k = \mathbf{x}_{k+1}-\mathbf{x}_{k},$$
where $\mathbf{J}(\mathbf{x}_k)$ is the Jacobian matrix of $\mathbf{F}$ evaluated at $\mathbf{x}_k$.
This approximation transforms the original non-linear optimization problem into a linear minimization problem:
$$
\mathbf{\delta}_{k}^{*}=\argmin_{\mathbf{\delta}_{k}}{||\mathbf{F}(\mathbf{x}_{k})+\mathbf{J}(\mathbf{x}_{k})\mathbf{\delta}_{k}||^2}.
$$
%where $\mathbf{\delta}_k^{*}$ is the optimal solution.
This is a highly over-constrained linear system for which we obtain the optimal least-squares solution $\mathbf{\delta}_k^{*}$ by solving the corresponding normal equations:
$$
2 \mathbf{J}(\mathbf{x}_{k})^T\mathbf{J}(\mathbf{x}_{k})\mathbf{\delta}_{k}^{*} = - 2\mathbf{J}(\mathbf{x}_{k})^T\mathbf{F}(\mathbf{x}_{k}).
$$
To solve the original non-linear minimization problem, we thus need to solve a sequence of linearized problems;
i.e., we initialize $\mathbf{x}_0$, and successively compute the update $\mathbf{\delta}_k^{*}$ from $\mathbf{x}_k$ to $\mathbf{x}_{k+1}$.
