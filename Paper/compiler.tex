\section{compiler}
\label{sec:compiler}

\JRK{Compiler \emph{in terms of what the solver needs (JTJ)}}

The \OPT compiler takes energy functions defined in our expression language and generates efficient, in-place Gauss-Newton solvers, in either blocked or non-blocked form, following the model described in the previous section.
\JRK{This proceeds in three steps: JTJp residual generation, code template expansion, and code generation???}

Our compiler, including the energy function expression language, is built using the metaprogramming and code-generation facilities of the Lua/Terra language environment \cite{terra}.

\subsection{Meta-solver templates} % (fold)
\label{sub:solver_templates}
In \OPT, solver methods are defined as metaprograms consisting of:

\begin{itemize}
  \item \emph{templates}, which define the core, application-independent logic to initialize and iterate the method, and
  \item \emph{generation logic}, which uses the compiler infrastructure to derive required and splice them into the template to produce an application-specific in-place solver.
\end{itemize}

\JRK{Solver templates implement XYZ metaprogramming interface/API}
% subsection solver_templates (end)

\subsection{Compiling Energy Functions to Residual Terms} % (fold)
\label{sec:jtj_generation}

The most complex logic in the compiler is responsible for generating code to compute the specific derivative terms in the form required by a given method from the original user-provided energy function.
In our parallel Gauss-Newton solvers, this is the set of partial derivatives corresponding to $J^TJp$ at a given unknown pixel $X(0,0)$. (Because of the stencil-only restriction on energy functions, these expressions are the same for every pixel, modulo boundary effects.)
In this section, we present the algorithm to derive this expression from an arbitrary energy function in the \OPT expression language, and show how it works when applied to a Laplacian smoothing energy on a 2D image.

% Cost is sum of squares, we keep each of those squared terms (residual r_i) separate and do the following for each i:
%% Making JTJp - we want it separately, for each unknown point - gather form per image pixel
% 1. Given cost, find the residuals that include x(0,0) - the non-zero entries in J^T
% 2. Unknowns for residual r(x,y) - the non-zero columns in J - just the stencil, shifted by x,y

\JRK{Understands data structures, stencils}
\JRK{How do shared block reads/writes affect this, if at al?}


\subsubsection{Automatic Differentiation}
\label{sec:autodiff}
Our compiler includes the ability to compute partial derivatives of , using a simplified variant of the \DSTAR symbolic differentiation algorithm \cite{dstar}. \JRK{Zach, what is actually worth saying about this? We treat everything as independent variables, no knowledge of stencils, etc.?}

\JRK{What we're doing: mixed numeric-symbolic differentiation. Applying autodiff transform to a symbolic representation at each unknown, but then compute and combine them for the whole system numerically.}

% subsection jtj_generation (end)

\subsection{Code Generation} % (fold)
\label{sub:codegen}

% subsection codegen (end)